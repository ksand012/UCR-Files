

\documentclass{article}

\usepackage{fullpage,latexsym,picinpar,amsmath,amsfonts}

\input{macros.tex}

\begin{document}
\centerline{REMOVED}
\centerline{REMOVED}
\centerline{\large \bf CS/MATH111 ASSIGNMENT 1}

\vskip 0.1in

\vskip 0.2in

%%%%%%%%%%%%%%%%%%%%%%%%%%%%

\begin{problem}
Let $W(n)$ be the number of
times ``whatsup" is printed by Algorithm~\textsc{WhatsUp} (see below) on input $n$.
Determine the asymptotic value of $W(n)$.


\begin{tabbing}
aa \= aa \= aa \= aa \= aa \= aa \= \kill
\textbf{Algorithm} \textsc{WhatsUp} $(n: \mbox{\bf integer})$ \\
      \> \textbf{for} $i \leftarrow 1$ \textbf{to} $2n$
                         \textbf{do} \\
      \> \> \textbf{for} $j \leftarrow 1$ \textbf{to} $(i+1)^2$ \textbf{do} \\
      \> \> \> print(``whatsup")
\end{tabbing}

Your solution must consist of the following steps:
%
\begin{description}
\item{(a)} First express $W(n)$ using summation notation $\sum$.
\item{(b)} Next, give a closed-form formula for $W(n)$. (A "closed-form formula"  
			should be a simple arithmetic expression without 
			any summation symbols.)
\item{(c)}  Finally, give the asymptotic value of $W(n)$ using the $\Theta$-notation. 
\end{description}
%
Show your work. Include a justification for each step. 

\smallskip
\noindent
\emph{Note:} If you need any summation formulas for this problem, you are allowed to
look them up.
\end{problem}

\begin{solution}
\begin{description}
\item{(a)} The algorithm \textsc{WhatsUp} goes through a nested for loop scenario. The first for loop goes through $2n$ times, while the inner for loop goes through $(i+1)^2$ times. This would result in producing a summation notation of $\sum_{i=1}^{2n} {(i+1)^2}$.

\item{(b)} With the summation notation known, we can now solve it below using the following formulas:
\smallskip
\begin{equation}
%
\frac{2n*(2n+1)(4n+1)}{6} \> \> \> \frac{4n(2n+1)}{2} \> \> \> 2(n)
\smallskip
\end{equation}
You then split the single summation into three separate summations and add them together:
\smallskip
\begin{equation}
\sum_{i=1}^{2n} {i^2} \> + \> \sum_{i=1}^{2n} {2i} \> + \> \sum_{i=1}^{2n} {1}
\end{equation}
\smallskip
We finally solve and simplify the 3 equations.
\begin{equation}
    = \frac{2n*(2n+1)(4n+1)}{6} \> + \> \frac{4n(2n+1)}{2} \> + \> 2(n)
\end{equation}
\smallskip
\begin{equation}
    = \frac{8n^3+18n^2+13n}{3}
\end{equation}

\item{(c)} With the answer to (b) solved, we can now figure out the asymptotic value of $W(n)$ using the $\Theta$-Notation:

\begin{equation}
    \frac{8n^3+18n^2+13n}{3} \le \frac{8n^3+18n^3+13n^3}{3}
\end{equation}


The original equation is less than or equal to all the values being raised to the highest power in the polynomial. We can now simplify the answer:
\smallskip
\begin{equation}
    =13n^3
\end{equation}
Therefore, the Big $O$ Notation is O($n^3$).



Now we need to find $\Omega$-Notation:

\smallskip
\begin{equation}
\frac{8n^3+18n^3+13n^3}{3} \ge \frac{8n^3+18n^2+13n}{3}
\end{equation}
\smallskip

The equation with the highest power in the polynomial is greater than or equal to all the values in the original polynomial. We can now simplify the answer:
\smallskip
\begin{equation}
    =13n^3
\end{equation}
Therefore, the Big $\Omega$-Notation Notation is $\Omega$($n^3$).

With these two figures in mind, we can determine that the Big $\Theta$ for this equation is $\Theta$$(n^3)$.

\end{description}
\end{solution}

%%%%%%%%%%%%%%%%%%%%%%%%%%%%
\setcounter{equation}{0}
\begin{problem}
Consider a sequence defined recursively as
$T_0 = 1$, $T_1 = 2$, and $T_n = T_{n-1}+3T_{n-2}$ for
$n\ge 2$. Prove that $T_n = O(2.4^n)$ and $T_n = \Omega(2.3^n)$.

\smallskip
\noindent
\emph{Hint:} 
First, prove by induction that $\half\cdot 2.3^n \le T_n \le 2.4^n$ for all $n\ge 0$.
%This is similar to the proof from class for the estimate of Fibonacci numbers.
\end{problem}

\begin{solution}
\begin{description}

To begin, we first need to prove the base case for $T_k \ge \frac{2.3^k}{2}$
\begin{equation}
    n = 0, T_0 = 1 \ge 1 = 2.3^0
\end{equation}
\begin{equation}
    n = 1, T_1 = 2 \ge \frac{2.3}{2} = 2.3^1
\end{equation}

The base cases prove true, so we move on to our induction step.

Induction:
\begin{equation}
T_k = T_{k-1} + 3T_{k-2} \> \> \> \> \textbf{for k} \ge 2
\end{equation}
\begin{equation}
T_k \ge \frac{2.3^{(k-1)}}{2} + 3 * 2.3^{(k-2)}
\end{equation}
\begin{equation}
= \frac{2.3^{(k-1)}}{2} * (3*2.3^{(-1)} + 1)
\end{equation}
\begin{equation}
= \frac{2.3^{(k-1)}}{2} * \frac{5.3}{2.3}
\end{equation}


We can replace $\frac{5.3}{2.3}$ with 2.3 because 2.3 $\le \frac{5.3}{2.3}$


\begin{equation}
\ge \frac{2.3^{(k-1)}}{2} * 2.3 = \frac{2.3^k}{2}
\end{equation}
\begin{description}
Thus $T_k \ge 2.3^n$ for all n $\ge 0$, $T_k = O(\frac{2.3^k}{2})$.
\end{description}

Next up, we show the second proof, where we need to prove the base case for $T_k \le 2.4^n$.

\begin{equation}
    n = 0, T_0 = 1 \le 1 = 2.4^0
\end{equation}
\begin{equation}
    n = 1, T_1 = 2 \le 2.4 = 2.4^1
\end{equation}

The base cases prove true, so we move on to our induction step.

Induction:
\begin{equation}
T_k = T_{k-1} + 3T_{k-2} \> \> \> \> \textbf{for k} \ge 2
\end{equation}
\begin{equation}
T_k \le 2.4^{(k-1)} + 3 * 2.4^{(k-2)}
\end{equation}
\begin{equation}
= 2.4^{(k-1)} * (3*2.4^{(-1)} + 1)
\end{equation}
\begin{equation}
= 2.4^{(k-1)} * \frac{5.4}{2.4}
\end{equation}


We can replace $\frac{5.4}{2.4}$ with 2.4 because 2.4 $\ge \frac{5.4}{2.4}$


\begin{equation}
\le 2.4^{(k-1)} * 2.4 = 2.4^k
\end{equation}
\begin{description}
Thus $T_k \le 2.4^n$ for all n $\ge 2$, $T_k = \Omega(2.4^k)$.
\end{description}
\begin{description}
Finally, we have proved that $\frac{1}{2} * 2.3^n \le T_n \le 2.4^n$ for all $n \ge 0$.
\end{description}

\end{description}
\end{solution}

%%%%%%%%%%%%%%%%%%%%%%%%%%%%

\begin{problem}
Give the asymptotic values of the
following functions, using the $\Theta$-notation:
%
\begin{description}
%
\item{(a)} $7n^2 + 2n^4 + 3n + 1$
\item{(b)} $5/n + \log_3 n + 11\sqrt{n}$
\item{(c)} $2n ( \log n + n^2) + 3n^4/\log n$
\item{(d)} $25n^{12} +  1.1^n + n^3\log^4 n$
\item{(e)} $n^72^n + 5\cdot 3^n$
%
\end{description}
%
Justify your answer.
(Here, you don't need to give a complete rigorous proof.
Give only an informal explanation using asymptotic
relations between the functions $n^c$, $\log n$, and $c^n$.)
\end{problem}

\begin{solution}
\begin{description}
\item{(a)} $f(n) = 7n^2 + 2n^4 + 3n + 1 = \theta(n^4)$

We have $O(n^2) + O(n^4) + O(n) + O(1)$

For $n \ge 1$

$O(1) \le O(n^4)$

$O(n) \le O(n^4)$

$O(n^2) \le O(n^4)$

So, $f(n) = O(n^4)$

$2n^4 \ge n^4$

So, $7n^2 + 2n^4 + 3n + 1 \ge n^4$

Then, $f(n) = \Omega(n^4)$

Therefore, 

$f(n) = \theta(n^4)$

\item{(b)} $f(n) = 5/n + \log_3 n + 11\sqrt{n} = \Theta(n^{1/2})$

For $n \ge 2$, 

$1/n \le n^{1/2}$, so $5/n = O(n^{1/2})$


Similarly, 

$\log_3 n \le n^{1/2}$ so $\log_3 n = O(n^{1/2})$

We have $f(n) = 5/n + \log_3 n + 11\sqrt{n}$ = $O(n^{1/2}) + O(n^{1/2}) + O(n^{1/2}) = O(n^{1/2})$

Now $11\sqrt{n} \ge \sqrt{n} = \Omega(n^{1/2})$

Therefore $5/n + \log_3 n + 11\sqrt{n} \ge \sqrt{n}$

So, $f(n) = \Omega(n^{1/2})$

Thus, 

$f(n) = \theta(n^{1/2})$

\item{(c)} $f(n) = 2n ( \log n + n^2) + 3n^4/\log n = \theta(n^4/\log n)$

$2n ( \log n + n^2) + 3n^4/\log n = 2n\log n + 2n^3 + 3n^4/\log n$

$(1/\log n)(2n\log^2 n + 2n^3\log n + 3n^4)$

For $n \ge 1$

$\log^2 n \le n^4$

$n^2 \le n^4$

So $f(n) = (1/\log n)(O(n^4) + O(n^4) + O(n^4)) = O(n^4/\log n)$

$2n ( \log n + n^2) + 3n^4/\log n \ge n^4/\log n$

So $f(n) = \Omega(n^4/\log n)$

Therefore, 

$f(n) = \theta(n^4/\log n)$


\item{(d)} $f(n) = 25n^{12} +  1.1^n + n^3\log^4 n = \theta(1.1^n)$

For $ n \ge 1$

$n^{12} \le 1.1^n$

$n^3\log^4 n \le 1.1^n$

So $25n^{12} +  1.1^n + n^3\log^4 n = O(1.1^n)$

$25n^{12} +  1.1^n + n^3\log^4 n \ge 1.1^n$

So $25n^{12} +  1.1^n + n^3\log^4 n = \Omega(1.1^n)$

Therefore,

$f(n) = \theta(1.1^n)$

\item{(e)} $f(n) = n^72^n + 5\cdot 3^n = \theta(3^n)$

$ = 2^n(n^7 + 1.5^n)$

For $n \ge 1$

$n^7 \le 1.5^n$

So $O(2^n)(O(1.5^n) + O(1.5^n)) = O(2^n)O(1.5^n)$

$= O(3^n)$

$n^72^n + 5\cdot 3^n \ge 3^n$

So $n^72^n + 5\cdot 3^n = \Omega(3^n)$

Therefore,

$f(n) = \theta(3^n)$

\end{description}
\end{solution}

%%%%%%%%%%%%%%%%%%%%%%%%%%%%

\vskip 0.1in
\paragraph{Submission.}
To submit the homework, you need to upload the pdf file into Gradescope (1 submission per group) and iLearn (each student has to submit individually). Late submissions will not be accepted.

\paragraph{Reminders.}
Remember that only papers created with {\LaTeX} are accepted. 

\end{document}

