\documentclass{article}

\usepackage{fullpage,latexsym,picinpar,amsmath,amsfonts}

\input{macros.tex}

\begin{document}
\centerline{REMOVED}
\centerline{REMOVED}
\centerline{\large \bf CS/MATH111 ASSIGNMENT 2}


\vskip 0.1in
%\noindent{\bf Individual assignment:} Problems 1 and 2.

%\noindent{\bf Group assignment:} Problems 1,2 and 3.

\vskip 0.2in

%%%%%%%%%%%%%%%%%%%%%%%%%%%%

\begin{problem}

Let $n = p_1 p_2 ... p_k$, where $p_1, p_2, ..., p_k$ are different primes. Prove that $n$ has exactly $2^k$ different divisors. 
For example, if $n = 105$, then n = $3 \cdot 5 \cdot 7$, so $k = 3$, and thus $n$ has $2^3 = 8$ divisors. These divisors are: $1, 3, 5, 7, 15, 21, 35, 105$.
Hint. You can reduce the problem to counting other objects that we already know how to count. Alternatively, this can be proved by induction on $k$.

\end{problem}

%%%%%%%%%%%%%%%%%%%%%%%%%%%%

\begin{solution}

The divisors of $n$ are $1$ and the numbers that can be made by taking the product of different combinations of the prime factors of $n$.
For example if the prime factors of n are $p_1, p_2,$ and $p_3$, then its divisors are $1, p_1, p_2, p_3, p_1p_2, p_1p_3, p_2p_3,$ and  $p_1p_2p_3$

\smallskip

If $k$ is the number of prime factors of $n$, we can find the number of divisors of $n$ by counting how many combinations of the primes factors of $n$ we can make using $1$ number up to $k$ numbers, plus $1$ because 1 is also a divisor of $n$. This can be written as:
$$\sum_{i=1}^k {k \choose i}+1$$
Because ${k \choose 0} = 1$ we can rewrite the above as:
$$\sum_{i=0}^k {k \choose i}$$
Recall the Binomial Theorem:
$$\sum_{i=0}^k {k \choose i}a^{k-i}b^{i} = (a+b)^{k}$$
If we apply the Binomial Theorem we get:
$$\sum_{i=0}^k {k \choose i}=\sum_{i=0}^k {k \choose i}(1)^{k-i}(1)^{i}=(1+1)^{k} = 2^{k}\>\>\>\>\>\>\>\>\>\>\>\>\>\>\square$$


\end{solution}

%%%%%%%%%%%%%%%%%%%%%%%%%%%%

\begin{problem}
Alice's RSA public key is $P = (e,n) = (13,77)$.
Bob sends Alice the message by encoding it as follows.
First he assigns numbers to characters:
A is 2, B is 3, ..., Z is 27, and blank is 28. Then he
uses RSA to encode each number separately. 

Bob's encoded message is:

\begin{verbatim}
     10       7      58      30      23      62 
      7      64      62      23      62      61 
      7      41      62      21       7      49 
     75       7      69      53      58      37 
     37      41      10      64      50       7 
     10      64      21      62      61      35 
     62      61      62       7      52      10 
     21      58       7      49      75       7 
     62      26      22      53      30      21 
     10      37      64
\end{verbatim}

Decode Bob's message.
Notice that you don't have Bob's secrete key, so you
need to ``break" RSA to decrypt his message.

\smallskip
For the solution, you need to provide the following:
%
\begin{itemize}
%
\item Describe step by step how you arrived at the solution.
	In particular, explain how you determined $p$, $q$, $\phi(n)$, and $d$.
%
\item Show the calculation that determines the first letter in the message from the first number in ciphertext.
%
\item Give Bob's message in plaintext. The message is a quote. Who said it?
%
\item If you wrote a program, attach your code to the hard copy.
	If you solved it by hand (not recommended), attach your scratch paper with calculations
	for at least 5 first letters.
%
\end{itemize}

Suggestion: this can be solved by hand, but it will probably
be faster to write a short program.
\end{problem}

\begin{solution}
We need to figure out Bob's secret key by "Breaking" RSA. This is done through guessing factorization of $n$. We first factor $n$. $n$ = 77, which can be broken up into $11 * 7$. These numbers are prime numbers. $p=7$ and $q = 11$.
\newline

We then compute Euler's Totient Function $φ(n)$:
\newline

$φ(n) = (p-1)(q-1) = (7-1)(11-1) = 6*10 = 60.$
\newline

We now compute the secret key $d$ by computing $e^{-1}(modφ(n))$ and solve by using the formula:
\newline

$d = e^{-1} (modφ(n))$

$= 13^{-1} (mod(60)) $
\newline

which turns into $13a = 60b + 1$
\newline

$13a = 60b + 1$
\newline

$0 = 8b (mod 13) + 1 $
\newline

$8b = -1 (mod 13) $
\newline

Find a multiple of 8 by adding 13 each time, which will result in getting 64.
\newline

$b = 64 / 8 = 8 $
\newline

Put b back into original equation, where b = 8
\newline

$13 a = 8 * 60 + 1 = 481$
\newline

$481 / 13 = 37 $
\newline

$d = 37$
\newline

Using the private key pair ($d, n$), where it is ($37, 77$) we can now decrypt messages by $M = C^{d} (mod(n))$ where C is the encrypted messages. So it would be: $M = C^{37} * mod(77)$
\newline

The calculation for the first letter in the message from the first number in ciphertext is:

The first number in ciphertext is 10.
\newline
10^{d}\pmod{n} = 10^{37}\pmod{77} \\
= (10^2)^{18}* 10\pmod{77} = 23^{18}* 10\pmod{77} \\
= (23^2)^{9}* 10\pmod{77} = 67^9* 10\pmod{77} \\
= (67^2)^{4}* 67* 10\pmod{77} = 23^4* 67* 10\pmod{77} \\ 
= (23^2)^{2}* 670\pmod{77} = 67^2* 670\pmod{77} \\ 
= 23* 670\pmod{77} = 15,410\pmod{77} = 10 \\ 
\newline
\textrm{The number that was decrypted from the ciphertext is 10, and based on the required formula, the character value is "I"}
\newline

After running the code, we get the following decrypted message from Bob: 
\newline

"I HAVE NEVER LET MY SCHOOLING INTERFERE WITH MY EDUCATION".
\newline
This is a quote by Mark Twain.
\newline

The code for solution 2 is on the last page of this assignment.

\newline

\end{solution}

%%%%%%%%%%%%%%%%%%%%%%%%%%%%

\begin{problem}
(a) Compute $13^{-1}\pmod{19}$ by enumerating multiples of the number and the modulus.
Show your work.

\smallskip\noindent
(b) Compute $13^{-1}\pmod{19}$ using Fermat's theorem. Show your work.

\smallskip\noindent
(c) Compute $13^{-40}\pmod{19}$ using Fermat's theorem. Show your work.

\smallskip\noindent
(d) Find a number $x\in\braced{1,2,...,36}$
	such that $8x \equiv 3 \pmod{37}$. Show your work.
	(You need to follow the method covered in class; brute-force checking
	all values of $x$ will not be accepted.)
\end{problem}

%%%%%%%%%%%%%%%%%%%%%%%%%%%%

\begin{solution}

\smallskip\noindent
(a) We must find an x and y that satisfy $13x + 19y = 1$
To solve this we can enumerate multiples of 13 and 19

\noindent
13: 13, 26, 39

\noindent
19: 19, 38

\noindent
We can see that $x = 3$ and $y = -2$ is a solution

\noindent
Therefore $13^{-1} = 3$ (mod 19)

\smallskip\noindent
(b) By Fermats Little Theorem: 

$a^{p-1} \equiv 1$ \pmod{p}

$a * a^{p-2} \equiv 1$ \pmod{p}

$a^{-1}$ (mod p) \equiv $a^{p-2}\pmod{p}$ 

$$13^{-1} \pmod{19} \equiv 13^{19-2} \equiv 13^{17} \pmod{19}$$
$$13 \equiv 13 \pmod{19}$$
$$13^{2} \equiv 17 \pmod{19}$$
$$13^{4} \equiv 13^{2} * 13^{2} \equiv 17^{2} \pmod{19} \equiv 4 \pmod{19}$$
$$13^{8} \equiv 13^{4} * 13^{4} \equiv 4^{2} \pmod{19} \equiv 16 \pmod{19}$$
$$13^{16} \equiv 13^{8} * 13^{8} \equiv (13^{8})^{2} \equiv 16^{2} \pmod{19} \equiv 9 \pmod{19}$$
$$13^{17} \equiv 13 * 13^{16} \equiv 13*9 \pmod{19} \equiv 117 \pmod{19} \equiv 3$$
$$13^{-1} \equiv 3 \pmod{19}$$

\smallskip\noindent
(c) We can rewrite $13^{-40}\pmod{19}$ as $(13^{40})^{-1}\pmod{19}$

\smallskip\noindent
Using Fermat's Theorem:
$$(13^{40})^{-1} = (13^{40})^{17} = 13^{680}\pmod{19}$$
$$13^{18} = 1 \pmod{19}$$
$$13^{680} = 13^{37*18+14} = (13^{18})^{37}*13^{14} = 13^{14}\pmod{19}$$
$$13^{14} = 13^{8+4+2} = 16*4*17 = 1088 = 5\pmod{19}$$
$$13^{-40} = 5\pmod{19}$$

\smallskip\noindent
(d) We can find $8^{-1}$ (mod 37) by enumerating multiples of 37 + 1

\smallskip\noindent
$8x\in\braced{1, 38, 75, 112, 149, ...}$

\smallskip\noindent
$8 * 14 = 112$ so 

\smallskip\noindent
$8^{-1} = 14$ (mod 37)

\smallskip\noindent
$x = 3 * 8^{-1}$ (mod 37) $ = 3*14 = 42 = 5$ (mod 37)

\end{solution}

%%%%%%%%%%%%%%%%%%%%%%%%%%%%
\vskip 0.1in
\paragraph{Submission.}
To submit the homework, you need to upload the pdf file into gradescope
by Friday, May 4 (noon).

%%%%%%%%%%%%%%%%%%%%%%%%%%%%%%%%%%%%%%%%%%%%%%%%%%%%%%%%%%%%%%%%%%%%%


\end{document}

