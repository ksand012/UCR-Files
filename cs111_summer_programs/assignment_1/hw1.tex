

\documentclass{article}

\usepackage{fullpage,latexsym,picinpar,amsmath,amsfonts}

\input{macros.tex}

\begin{document}
\centerline{REMOVED}
\centerline{REMOVED}
\centerline{\large \bf CS/MATH111 ASSIGNMENT 1}
%\centerline{due : Friday, April 20 (noon)}

\vskip 0.1in
%\noindent{\bf Individual assignment:} Problems 1, 2, and 3.

%\noindent{\bf Group assignment:} Problems 1,2 and 3.

\vskip 0.2in

%%%%%%%%%%%%%%%%%%%%%%%%%%%%
\large
\begin{problem}
Let $W(n)$ be the number of
times ``whatsup" is printed by Algorithm~\textsc{WhatsUp} (see below) on input $n$.
Determine the asymptotic value of $W(n)$.


\begin{tabbing}
aa \= aa \= aa \= aa \= aa \= aa \= \kill
\textbf{Algorithm} \textsc{WhatsUp} $(n: \mbox{\bf integer})$ \\
      \> \textbf{for} $i \leftarrow 1$ \textbf{to} $n^2-3$
                         \textbf{do} \\
      \> \> \textbf{for} $j \leftarrow 1$ \textbf{to} $(i-1)^2$ \textbf{do} \\
      \> \> \> print(``whatsup")
\end{tabbing}

Your solution must consist of the following steps:
%
\begin{description}
\item{(a)} First express $W(n)$ using summation notation $\sum$.
\item{(b)} Next, give a closed-form formula for $W(n)$. (A "closed-form formula"  
			should be a simple arithmetic expression without 
			any summation symbols.)
\item{(c)}  Finally, give the asymptotic value of $W(n)$ using the $\Theta$-notation. 
\end{description}
%
Show your work. Include a justification for each step. 

\smallskip
\noindent
\emph{Note:} If you need any summation formulas for this problem, you are allowed to
look them up.
\end{problem}

\Large
\begin{solution}
\begin{description}
\item{(a)} The algorithm \textsc{WhatsUp} goes through a nested for loop scenario. The first for loop goes through $n^2-3$ times, while the inner for loop goes through $(i-1)^2$ times. This would result in a summation notation of: \\$\sum_{i=1}^{n^2-3} {(i-1)^2}$.
\item{(b)}This can then be reduced through the following steps to give the closed-form formula:
\newline
\\First expand the internal expression:
\\$\sum_{i=1}^{n^2-3}(i^2-2i+2) =$ 
\newline
\\Then break up the expansion:
\\$\sum_{i=1}^{n^2-3}(i^2) + \sum_{i=1}^{n^2-3}(-2i) + \sum_{i=1}^{n^2-3}(2) =$
\newline
\\Next reduce the expression
\\$\sum_{i=1}^{n^2-3}(i^2) -2\sum_{i=1}^{n^2-3}(i) + 2(n^2 - 3) =$
\newline
\\Then set up the sums to be turned into algebraic expressions
\\$\sum_{i=1}^{n^2-3}(i^2) -2\sum_{i=1}^{n^2-3}(i) + 2n^2 - 6$=
\newline
\\Next use the first expression listed below to turn a sum into an algebraic expression
\\ $\sum_{i=1}^{n^2-3}(i^2) - 2(\frac{(n^3-3)(n^2-2)}{2}) + 2n^2 - 6 =$
\newline
\\Then use the second expression below to turn the remaining sum into an algebraic expression
\\$\frac{(n^2-3)(n^2-2)(2(n^2-3))}{6} - 2(\frac{(n^3-3)(n^2-2)}{2}) + 2n^2 - 6 =$
\newline
\\Then simplify
\\ $\frac{(n^4-5n^2+5)(2n^2-6)}{6} - \frac{2n^4+10n^2-10}{2} + 2n^2 - 6 =$
\newline
\\Then further simplify
\\ $\frac{2n^6-16n^4+40n^2-30}{6} - n^4+5n^2-5+2n^2-6 =$
\newline
\\Next continue simplifying the expression
\\ $\frac{1}{3}n^6 - \frac{8}{3}n^4 + \frac{20}{3}n^2 - 5 - n^4+5n^2-5+2n^2-6 =$
\newline
\\Simplify again
\\ $\frac{1}{3}n^6 - \frac{11}{3}n^4 + \frac{35}{3}n^2 + 2n^2 -16 =$
\newline
\\Finally reduce to the final expression
\\ $\frac{1}{3}n^6 - \frac{11}{3}n^4 + \frac{41}{3}n^2 - 16$
\\\\Note that this formula is derived using two basic arithmetic formulas to change a sum of something into an equation.  These formulas are listed below:
\\ $\sum_{k=1}^{n}(k) = \frac{n(n+1)}{2}$
\\ $\sum_{i=1}^{n}(k^2) = \frac{n(n+1)(2n+1)}{6}$
\\\\These formulas can be seen utilized in steps five and six above.
\newline
\item{(c)}This expression from (b) can be used to solve for the Big $\Theta$ value:
\\ $\frac{1}{3}n^6 - \frac{11}{3}n^4 + \frac{41}{3}n^2 - 16 = O(n^6)$
\newline
\\ for all $n>=2$:
\\ $\frac{1}{3}n^6 - \frac{11}{3}n^4 + \frac{41}{3}n^2 - 16 >= \frac{1}{3}n^6$
\newline
\\Which gives the Big- $\Omega$ value:
\\ Therefore $\frac{1}{3}n^6 - \frac{11}{3}n^4 + \frac{41}{3}n^2 - 16 = \Omega (n^6)$
\newline
\\ Therefore $\frac{1}{3}n^6 - \frac{11}{3}n^4 + \frac{41}{3}n^2 - 16 = \Theta (n^6)$

\end{description}
\end{solution}

%%%%%%%%%%%%%%%%%%%%%%%%%%%%

\begin{problem}
Consider a sequence defined recursively as
$T_0 = 1$, $T_1 = 2$, and $T_n = T_{n-1}+3T_{n-2}$ for
$n\ge 2$. Prove that $T_n = O(2.5^n)$ and $T_n = \Omega(2.25^n)$.

\smallskip
\noindent
\emph{Hint:} 
First, prove by induction that $\half\cdot 2.25^n \le T_n \le 2.5^n$ for all $n\ge 0$.
%This is similar to the proof from class for the estimate of Fibonacci numbers.
\end{problem}

\begin{solution}
%Solution goes here.
\Large
\begin{proof}
\\\\Big-O
\\\\Let $T_n=T+3T_{n-2}$ for all $n>2, T_0=1, T_1=2$
\\\\Prove $T_n=0(2.5^n)$ such that $T_{k+1}<=2.5^{k+1}$
\\\\Base cases:
\\Here it is shown that the values of 1 and 2 for T are correct when compared to the value that will be proven below, and is shown above
\\$T_0=1<2.5^0$
\\$T_1=2<=2.5^1$
\\\\Assumption:
\\Here the above value to be proven is expanded to k+1 to be used in the induction step
\\$T_{k+1}<=2.5^{k+1}$ for all $n>=0$
\\\\Induction:
\\Here the values to be proven, shown above, are set in the original formula using the assumption above
\\$T_{k+1}=T_k+3T_{k+1}=2.5^k+3(2.5^{k-1})$
\endline
\\This equation is then simplified
\\\\$2.5^k+3(2.5^{k+1}) <= 2.5^{k+1}$
\\$2.5^k(1 + 3(2.5^{-1})) <= 2.5^{k+1}$
\\$2.5^k(1 + \frac{3}{2.5}) <= 2.5^{k+1}$
\\$2.5^k(\frac{2.5}{2.5} + \frac{3}{2.5}) <= 2.5^{k+1}$
\\$2.5^k(\frac{5.5}{2.5}) <= 2.5^{k+1}$
\\$2.5^k(2.5*\frac{5.5}{6.25}) <= 2.5^{k+1}$
\\$2.5^{k+1}(\frac{5.5}{6.25}) <= 2.5^{k+1}$
\endline
\\The final form of this equation is then seen to be true
\\\\Thus $T_{k+1} <= 2.5^{k+1}$ and $T_n = O(2.5^n)$
\\\\Big-\Omega
\\\\Let $T_n=T+3T_{n-2}$ for all $n>2, T_0=1, T_1=2$
\\\\Prove $T_n>=\frac{1}{2}*2.25^n$
\\\\Base cases:
\\Here it is shown that the values of 1 and 2 for T are correct when compared to the value that will be proven below, and is shown above
\\$T_0=1 >= \frac{2.25^0}{2}$
\\$T_1=2 >= \frac{2.25^1}{2}$
\\\\Assumption:
\\Here the above value to be proven is expanded to k+1 to be used in the induction step
\\$T-{k+1} >= \frac{2.25^{k+1}}{2}$ for all $n>=0$
\\\\Induction:
\\Here the values to be proven, shown above, are set in the original formula using the assumption above
\\$T_{k+1}=T_k+3T_{k+1}=2.5^k+3(2.5^{k-1})$
\endline
\\This equation is then simplified
\\$\frac{2.25^k}{2}+3(\frac{2.25^{k-1}}{2}) >= \frac{2.25^{k+1}}{2}$
\\$\frac{2.25^k}{2}(1+3(\frac{2.25^-1}{2})) >= \frac{2.25^{k+1}}{2}$
\\$\frac{2.25^k}{2}(1.667) >= \frac{2.25^{k+1}}{2}$
\\ $\frac{2.25^k}{2}(\frac{2.25}{2})(1.48) >= \frac{2.25^{k+1}}{2}$
\endline
\\The final form of this equation is then seen to be true
\\ $\frac{2.25^{k+1}}{2}(1.48) >= \frac{2.25^{k+1}}{2}$
\\\\Thus $T_{k+1} >= \frac{1}{2}*(2.25^{k+1})$ and $T_n = O(\frac{1}{2}2.25^n)$

\end{proof}

\end{solution}

%%%%%%%%%%%%%%%%%%%%%%%%%%%%

\begin{problem}
Give the asymptotic values of the
following functions, using the $\Theta$-notation:
%
\begin{description}
%
\item{(a)} $7n^2 + 2n^4 + 3n + 1$
\item{(b)} $5/n + 1/\log_3 n + 11/\sqrt{n}$
\item{(c)} $2n ( n\log n + n^2) + 3n^4/\log n$
\item{(d)} $25n^{12}  + n^3\log^4 n +  1.25^n$
\item{(e)} $n4^n + 5n^6\cdot 3^n$
%
\end{description}
%
Justify your answer.
(Here, you don't need to give a complete rigorous proof.
Give only an informal explanation using asymptotic
relations between the functions $n^c$, $\log n$, and $c^n$.)
\end{problem}

\begin{solution}
\begin{description}
\item{(a)} $f(n) = 7n^2 + 2n^4 + 3n + 1$
\newline
We have $O(n^2) + O(n^4) + O(n) + O(1)$ when disregarding the constants.

\newline

Suppose we have $n \ge 1$

\newline

Then the following would be true:
\newline
\newline
$O(1) \le O(n^4)$

$O(n) \le O(n^4)$

$O(n^2) \le O(n^4)$

So, $f(n) = O(n^4)$
\newline

We then produce $2n^4 \ge n^4$

So, $7n^2 + 2n^4 + 3n + 1 \ge n^4$
\newline
Then, $f(n) = \Omega(n^4)$
\newline
Therefore, we get $f(n) = \theta(n^4)$
\newline

\item{(b)} $f(n) = 5/n + \frac{1}{\log_3 n} + 11\sqrt{n}$
\newline
Suppose we have $n \ge 2$,
\newline
We want to work with the $\sqrt{n}$
\newline
$1/n \le n^{1/2}$, so $5/n = O(n^{1/2})$
\newline

We can also work with $\log_3 n$:
\newline
$\log_3 n \le n^{1/2}$ so $\log_3 n = O(n^{1/2})$
\newline

We now have $f(n) = 5/n + \log_3 n + 11\sqrt{n}$ = $O(n^{1/2}) + O(n^{1/2}) + O(n^{1/2}) = O(n^{1/2})$
\newline

We then get $11\sqrt{n} \ge \sqrt{n} = \Omega(n^{1/2})$
\newline

$5/n + \log_3 n + 11\sqrt{n} \ge \sqrt{n}$
\newline

We are left with $f(n) = \Omega(n^{1/2})$
\newline

Therefore $f(n) = \theta(n^{1/2})$

\item{(c)} $f(n) = 2n ( n\log n + n^2) + 3n^4/\log n$
\newline

We can distribute the $2n$ to expand the equation.
\newline
$2n ( n\log n + n^2) + 3n^4/\log n $
\newline
= $2n^2\log n + 2n^3 + 3n^4/\log n$

\newline
We can then pull out a $(1/\log n)$ from the expanded equation:
\newline
$(1/\log n)(2n^2\log^2 n + 2n^3\log n + 3n^4)$

Now, suppose we have $n \ge 1$

$2n^2\log^2 n \le n^4$
\newline
$n^2 \le n^4$
\newline

So $f(n) = (1/\log n)(O(n^4) + O(n^4) + O(n^4))$
\newline
$=O(n^4/\log n)$
\newline
$2n (n\log n + n^2) + 3n^4/\log n \ge n^4/\log n$
\newline
We are left with $f(n) = \Omega(n^4/\log n)$
\newline

Therefore $f(n) = \theta(n^4/\log n)$
\newline

\item{(d)} $f(n) = 25n^{12} + n^3\log^4 n + 1.25^n $
\newline

Suppose we have $ n \ge 1$
\newline
$n^{12} \le 1.25^n$
\newline
$n^3\log^4 n \le 1.25^n$
\newline

Based on above, we can show that $25n^{12} + n^3\log^4 n + 1.25^n = O(1.25^n)$
\newline
$25n^{12} + n^3\log^4 n + 1.25^n \ge 1.25^n$

Because it's larger than $1.25^n$, we have 
\newline
$25n^{12} + n^3\log^4 n + 1.25^n = \Omega(1.25^n)$
\newline

Therefore $f(n) = \theta(1.25^n)$
\newline

\item{(e)} $f(n) = n4^n + 5n^6 \cdot 3^n$
\newline

Because we cancel out constants, $5n^6$ converts to $n^6$.
\newline

When comparing $3^n$ and $n^6$:
\newline
$O(3^n) \ge O(n^6)$
\newline

Based on this, we can see that $ n4^n + 5n^6 \cdot 3^n \le n4^n + 5n^7 \cdot 3^n$
\newline

We can then pull out an n.
\newline
$n(4^n+5n^6\cdot3^n) = O(n) \cdot O(4^n)$
\newline
$=O(n4^n)$
\newline

Suppose we have $n^6 \le n4^n$
\newline
So $O(3^n \cdot n)((O(4/3)^n + (O)(4/3)^n) = O(n)O(4^n)
\newline
$= O(n4^n)$
\newline
With this in mind, we can find Big $\Omega$ through the following:

$n4^n + 5n^6 \cdot 3^n \ge n4^n$

We then have n4^n + 5n^6 \cdot 3^n = $\Omega$ $(n4^n)$

Therefore, $f(n) = \theta(n4^n)$

\end{description}

\end{solution}

%%%%%%%%%%%%%%%%%%%%%%%%%%%%

\vskip 0.1in
\paragraph{Submission.}
To submit the homework, you need to upload the pdf file into Gradescope (1 submission per group) and iLearn (each student has to submit individually). Late submissions will not be accepted.

\paragraph{Reminders.}
Remember that only papers created with {\LaTeX} are accepted. 



\end{document}

