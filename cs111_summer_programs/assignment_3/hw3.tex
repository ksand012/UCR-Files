\documentclass{article}

\usepackage{fullpage,latexsym,picinpar,amsmath,amsfonts,graphicx}

\input{macros.tex}

\begin{document}
\centerline{REMOVED}
\centerline{REMOVED}
\centerline{\large \bf CS/MATH111 ASSIGNMENT 3}
%\centerline{due Monday, February 25 (11:50PM)}

\vskip 0.2in
%\noindent{\bf Individual assignment:} Problems 1 and 2.

%\noindent{\bf Group assignment:} Problems 1 and 2.

\vskip 0.2in

%%%%%%%%%%%%%%%%%%%%%%%%%%%%

\begin{problem}
Solve the following recurrence equation:
%
\begin{eqnarray*}
        A_n &=& A_{n-1} + 2A_{n-2} + 3^n\\
        A_0 &=& 0 \\
        A_1 &=& 4
\end{eqnarray*}
%
Show your work (all steps: the associated homogeneous equation,
the characteristic polynomial and its
roots, the general solution of the homogeneous
equation, computing a particular solution,
the general solution of the non-homogeneous equation,
using the initial conditions to compute the final solution.)
\end{problem}
\newline

\begin{solution}
Associated homogeneous solution: $A_{n} = A_{n-1} + 2A_{n-2}$
\newline
Characteristic polynomial: $A_n - A_{n-1} - 2A_{n-2} = 0$
\newline
$A^2-A-2=0$ = $(A-2)(A+1)=0$
\newline
Roots: A=2, A=-1
\newline
General solution of homogeneous equation:
$\alpha_1 \cdot (2)^n + \alpha_2 \cdot (-1)^n = A'_n$
\newline

According to the slides, the form that a particular solution takes if the number is not a root of the function is, in our case, $c3^n$.
\newline

We then solve for c to know what exactly our particular solution is:
\newline

$c3^{(n-1)}+2c3^{(n-2)}+3^n = c3^n$
\newline

We can then simply from here:
\newline

$3c + 2c + 9 = 9c$
\newline

$c = \dfrac{9}{4}$
\newline

So, plugging it back in to the original formula, our particular solution becomes:
\newline

$A'_n = (\dfrac{9}{4}) \cdot 3^n$
\newline

Now that we have both our general homogeneous solution and our particular solution, we can add them together to get the general non-homogeneous solution from the formula $f_n = f'_n + f''_n$
\newline

$\alpha_1 \cdot (2)^n + \alpha_2 \cdot (-1)^n + (\dfrac{9}{4}) \cdot 3^n$
\newline

We now have all the information we need to solve for the final answer to the non-homogeneous solution:
\newline

$\alpha_1 \cdot (2)^0 + \alpha_2 \cdot (-1)^0 + (\dfrac{9}{4}) \cdot 3^0 = 0$
$\alpha_1 + \alpha_2 + (\dfrac{9}{4})$
\newline

$\alpha_1 \cdot (2)^1 + \alpha_2 \cdot (-1)^1 + (\dfrac{9}{4}) \cdot 3^1 = 4$
$2\alpha_1 - \alpha_2 + (\dfrac{27}{4}) = 4$
\newline

Now we can cancel out the $\alpha_2$'s. We are left with:
\newline
$3\alpha_1 + 9 = 4$
\newline
$\alpha_1 = \dfrac{-5}{3}$
\newline
$\alpha_2 = \dfrac{5}{3} - \dfrac{9}{4}$
\newline
$\alpha_2 = \dfrac{-7}{12}$
\newline

We plug these values back in to the starting non-homogeneous solution to get the final solution of:
\newline
$A_n = (\dfrac{-5}{3}) \cdot 2^n - (\dfrac{7}{12}) \cdot (-1)^n + (\dfrac{9}{4}) \cdot 3^n$

\end{solution}

%%%%%%%%%%%%%%%%%%%%%%%%%%%%
\vskip 0.3in

\begin{problem}
Give the asymptotic value (using the $\Theta$-notation)
for the number of letters that will be printed by the algorithms below.
In each algorithm the argument $n$ is a positive integer.
Your solution needs to consist of an appropriate recurrence 
equation and its solution. You also need to give a brief justification for
the recurrence. 

\bigskip
\noindent
Part (a) 
\vskip 0.2in
\noindent(i)\ \ 
\begin{minipage}[t]{3in}
\begin{tabbing}
aaa \= aaa \= aaa \= aaa \=  \kill
\textbf{Algorithm} \textsc{PrintAs} $(n: \mbox{\bf integer})$ \\
          \> \textbf{if} $n < 4$ \\
          \>\>  print(``A") \\
          \>\textbf{else} \\
          \>\>  \textsc{PrintAs}$(\ceiling{n/3})$\\
          \>\>  \textsc{PrintAs}$(\ceiling{n/3})$\\
          \>\>  \textsc{PrintAs}$(\ceiling{n/3})$\\
          \>\>  \textsc{PrintAs}$(\ceiling{n/3})$\\
           \>\>  \textsc{PrintAs}$(\ceiling{n/3})$\\
      \>\> \textbf{for} $i \leftarrow 1$ \textbf{to} $4n^2$ \textbf{do} print(``A")
\end{tabbing}
\end{minipage}

\vskip 0.2in
\noindent
(ii)\ \
\begin{minipage}[t]{3in}
\begin{tabbing}
aaa \= aaa \= aaa \= aaa \=  \kill
\textbf{Algorithm} \textsc{PrintBs} $(n: \mbox{\bf integer})$ \\
          \> \textbf{if} $n < 2$ \\
          \>\>  print(``B") \\
          \>\textbf{else} \\
          \>\>  \textbf{for} $j \leftarrow 1$ \textbf{to} $8$ 
					\textbf{do} \textsc{PrintBs}$(\floor{n/2})$\\
      \>\> \textbf{for} $i \leftarrow 1$ \textbf{to} $10n^3$ \textbf{do} print(``B")
\end{tabbing}
\end{minipage}

\vskip 0.2in
\noindent
(iii)\ \ 
\begin{minipage}[t]{3in}
\begin{tabbing}
aaa \= aaa \= aaa \= aaa \=  \kill
\textbf{Algorithm} \textsc{PrintCs} $(n: \mbox{\bf integer})$ \\
          \> \textbf{if} $n < 3$ \\
          \>\>  print(``C") \\
          \>\textbf{else} \\
          \>\>  \textsc{PrintCs}$(\ceiling{n/2})$\\
          \>\>  \textsc{PrintCs}$(\ceiling{n/2})$\\
          \>\>  \textsc{PrintCs}$(\ceiling{n/2})$\\
          \>\>  \textsc{PrintCs}$(\ceiling{n/2})$\\
      \>\> \textbf{for} $i \leftarrow 1$ \textbf{to} $20$ \textbf{do} print(``C")
\end{tabbing}
\end{minipage}

\vskip 0.3in
\noindent
Part (b)
\vskip 0.2in
(iv)\ \
\begin{minipage}[t]{3in}
\begin{tabbing}
aaa \= aaa \= aaa \= aaa \=  \kill
\textbf{Algorithm} \textsc{PrintDs} $(n: \mbox{\bf integer})$ \\
          \> \textbf{if} $n < 2$ \\
          \>\>  print(``D") \\
          \>\textbf{else} \\
          \>\>  \textbf{for} $j \leftarrow 1$ \textbf{to} $4$ 
					\textbf{do} \textsc{PrintDs}$(\floor{n/2})$\\
      \>\> \textbf{for} $i \leftarrow 1$ \textbf{to} $10n^k$ \textbf{do} print(``D"),
\end{tabbing}
%\vskip 0.1in     
      where $k$ is a nonnegative integer.

\end{minipage}
\end{problem}


\begin{solution}

\\\\(a i)
\\\\ There are 5 recursive calls in this function of $\frac{n}{3}$
\\\\ $4n^2$ is called at the end
\\\\ This gives a formula of:
\\ $x(n) = 5x(\frac{n}{3})+4n^2$
\\\\ This gives values of:
\\ $a=5, b=3, c=4, d=2$
\\ $b^d = 3^2=9$
\\ $5 < 9$ Therefore:
\\ $T(n) = \Theta(n^d) = \Theta(n^2)$
\newline
\\\\(a ii)
\\ There are 8 recursive calls in this function of $\frac{n}{2}$
\\ $10n^3$ is called at the end
\\\\ This gives a formula of:
\\ $x(n) = 8x(\frac{n}{2})+10^3$
\\\\ This gives values of:
\\ $a=8, b=2, c=10, d=3$
\\ $b^d = 2^3 = 8$
\\ $8=8$ Therefore:
\\ $T(n) = \Theta(n^dlog(n))$
\\ T(n) = \Theta(n^3log(n))
\newline
\\\\ $(a iii)$
\\\\ There are 4 recursive calls in this function of $\frac{n}{2}$
\\\\ $20$ is called at the end
\\\\ This gives a formula of:
\\ $x(n) = 4x(\frac{n}{2})+20$
\\\\ This gives values of:
\\ $a=4, b=2, c=20, d=0$
\\ $b^d = 2^0 = 1$
\\ $4>1$ Therefore:
\\ $T(n) = \Theta(n^{log_b a}) = \Theta(n^{log_2 4}) = \Theta(n^2)$
\newline
\\\\ (b iv)
\\\\ There are 4 recursive calls in this function of $\frac{n}{2}$
\\\\ $10n^k$ is called at the end
\\\\ This gives a formula of:
\\ $x(n) = 4x(\frac{n}{2})+10n^k$
\\\\ This gives values of:
\\ $a=4, b=2, c=10, d=k$
\\ $b^d = 2^k$
\\\\ This gives three conditions:
\\ if $ k <= 1: \Theta(n^{log_2 k})$
\\ if $k=2: \Theta(n^2log(2))$
\\ if $k >= 3: \Theta(n^k)$

\end{solution}

%%%%%%%%%%%%%%%%%%%%%%%%%%%%
\vskip 0.4in
\begin{problem}
Bill is buying his wife a bouquet of carnations, roses, tulips, daises, and lilies. 
The bouquet will have $28$ flowers, with 
%
\begin{itemize} 
		\item at most $6$ carnations, 
		\item between $3$ and $7$ roses,
        \item between $2$ and $15$ tulips,
        \item at most $4$ daises, and
    	\item at least $1$ lily.

\end{itemize}
%
How many different combinations of flowers satisfy these requirements?
You need to use the counting method for integer partitions and show your work.
%\end{problem}

\begin{solution}
\newline
R = roses
\newline
T = tulips
\newline
L = lilies
\newline
C = carnations
\newline
D = Daisies
\newline


\\\\ R, T, D, L, C, total of: 28
\newline

\\ $c \le 6$
\newline
\\ $3 \le R \le 7$ \xrightarrow{$R = R' + 3$}\xrightarrow{$R' = R - 3$}
\newline
\\ $2 \le T \le 15$ \xrightarrow{$T = T' + 2$}\xrightarrow{$T' = T - 2$}
\newline
\\ $D \le 4$
\newline
\\ $1 \le L$ \xrightarrow{$L = L' + 1$}\xrightarrow{$L' = L - 1$}
\newline
\\ $C+T+D+L+R = 28$
\newline
\\ $C+R'+T'+D+L' = 22$
\newline

\\ $0 \le C \le 6$
\\ $0 \le R' \le 4$
\\ $0 \le T' \le 13$
\\ $0 \le D \le 4$
\\ $0 \le L'$
\newline

\\ $S = ({28+(5-1) \choose (5-1)}) = ({32 \choose 4}) = \frac{32!}{(32-4)! \cdot 4!} = \frac{32*31*30*29*28!}{4!*28!} = 35,960$
\newline

\\ $S_{total} - S(c\ge7 \cup R' \ge 8 \cup T' \ge 16 \cup D \ge 5 \cup L' \le 0)$
\newline

\\\\ This can then be broken up into:
\\ $S_{total}-S(S(C \ge 7)+S(R' \ge 8)+S(T' \ge 16)+S(D \ge 5)+S(L' \le 0) -S(c \ge \cap R' \ge 8)- S(C \ge 7 \cap T' \ge 16)- S(C \ge 7 \cap D \ge 5)- S(C \ge 7 \cap L' \le 0)- S(R' \ge 8 \cap T' \ge 16)- S(R' \ge 8 \cap D \ge 5)- S(R' \ge 8 \cap L' \le 0)- S(T' \ge 16 \cap D \ge 5)- S(T' \ge 16 \cap L' \le 0)- S(D \ge 5 \cap L' \le 0) + S(c \ge 7 \cap R' \ge 8 \cap T' \ge 16 \cap d \ge 5 \cap L' \le 0))$
\newline

\\\\ This can then be turned into:
\\ $35,960 - (({22-7+4 \choose (5-1)}) + ({22-8+4 \choose (5-1)}) + ({22-16+4 \choose (5-1)}) + ({22-5+4 \choose (5-1)}) + ({22-0+4 \choose (5-1)}) - ({22-7-8+4 \choose (5-1)}) - ({22-7-16+4 \choose (5-1)}) - ({22-7-5+4 \choose (5-1)}) + ({22-7-0+4 \choose (5-1)}) - ({22-8-16+4 \choose (5-1)}) - ({22-8-5+4 \choose (5-1)}) - ({22-8-0+4 \choose (5-1)}) ({22-16-5+4 \choose (5-1)}) - ({22-16-0+4 \choose (5-1)}) - ({22-5-0+4 \choose (5-1)}) + ({22-8-5-16-7-0+4 \choose (5-1)}))$
\newline

\\\\ This reduces to:
\\ $35,960 - ( {19 \choose 4} + {18 \choose 4} + {10 \choose 4} + {21 \choose 4} + {26 \choose 4} - {11 \choose 4} - {3 \choose 4} - {14 \choose 4} - {19 \choose 4} - {2 \choose 4} - {13 \choose 4} - {18 \choose 4} - {5 \choose 4} - {10 \choose 4} - {21 \choose 4} + {0 \choose 4} )$
\newline

\\\\ This gives:
\\ $35,960 - ((\frac{19!}{(19-4)4!}) + (\frac{18!}{(18-4)4!}) + (\frac{10!}{(10-4)4!}) + (\frac{21!}{(21-4)4!}) + (\frac{26!}{(26-4)4!}) - (\frac{11!}{(11-4)4!}) - (\frac{3!}{(3-4)4!}) - (\frac{14!}{(14-4)4!}) - (\frac{19!}{(19-4)4!}) - (\frac{2!}{(2-4)4!}) - (\frac{13!}{(13-4)4!}) - (\frac{18!}{(18-4)4!}) - (\frac{5!}{(5-4)4!}) - (\frac{10!}{(10-4)4!}) - (\frac{21!}{(21-4)4!}) - (\frac{0!}{(0-4)4!}))$
\newline

\\\\ Which is:
\\ $35,960 - (3876 + 3060 + 210 + 5985 + 14950 - 330 - 0 - 1001 -3876 - 0 - 715 - 3060 - 5 - 210 - 5985 + 0) =$
\\ $35,960 - 12,899 = 23,061$
\newline

\\\\ So there are $23,061$ different combinations. 

\end{solution}



%%%%%%%%%%%%%%%%%%%%%%%%%%%%
\bigskip
\begin{problem}
\noindent 
We have three sets $P$, $Q$, $R$
with the following properties:

\begin{description}

\item{(a)}  $|P| = 3|Q|$ and 
                $|R| = |P|$,

\item{(b)} $|P\cap Q| = 11$,
        $ |Q\cap R|  = 14$,
        $ |P\cap R| = 2|Q|$,

\item{(c)}
$7\le |P\cap Q\cap R| \le 17$,

\item{(d)}
$|P\cup Q\cup R| = 100$.

\end{description}

Use the inclusion-exclusion principle to
determine the number of elements in $P$.
Show your work.
(Hint: You may get an equation with two unknowns, but one of them has only a few possible values.)
\end{problem}


\end{problem}

\begin{solution}

\\\\ Using the inclusion exclusion formula:
\\ $|A \cup B \cup C| = |A| + |B| + |C| - |A \cap B| - |A \cap C| - |B \cap C| + |A \cap B \cap C|$
\\\\ With the given values the below equation can be created:
\\ $100 = 3|Q| + 3|Q| + |Q| - 11 - 14 - 2|Q| + |P \cap Q \cap R|$
\\\\ This can be reduced to:
\\ $100 = 5|Q| - 25 + |P \cap Q \cap R|$
\\ $25 = |Q| + \frac{|P \cap Q \cap R|}{5}$
\\\\ This means that the $|P \cap Q \cap R|$ value must be evenly divisible by 5 as there isn't a fraction of elements possible.
\\\\ The options of numbers that could be this value are:
\\ $7, 8, 9, 10, 11, 12, 13, 14, 15, 16, 17$
\\\\ Of these values only 10 and 15 are divisible by five so one of those two values must be correct.
\\\\ This can be decided with the given: $|P| = 3|Q|$
\\\\ This means $|Q|$ must be a value evenly divisible by 3 to get a whole number of values for $|P|$ as $\frac{|P|}{3} = |Q|$
\\\\ This value must satisfy both $25 = |Q| + \frac{|P \cap Q \cap R|}{5}$ and $\frac{|P|}{3} = |Q|$ with whole numbers
\\\\ If $|P \cap Q \cap R| = 10$:
\\ Then $|Q| = 23$
\\ Solved this gives a $|P|$ of 69.

\\\\ If $|P \cap Q \cap R| = 15$:
\\ Then $|Q| = 22$
\\ Solved this gives a $|P|$ of 66.
\newline

\\\\ Of these two the value that holds 15 for $|P \cap Q \cap R|$ can be discounted as it falls outside the bounds of $|P \cup Q|$ and $|Q \cup R|$
\\\\ Therefore $|P|$ must be 69.
\end{solution}

%%%%%%%%%%%%%%%%%%%%%%%%%%%%

\vskip 0.1in
\paragraph{Submission.}
To submit the homework, you need to upload the pdf file into ilearn and Gradescope.


\end{document}

